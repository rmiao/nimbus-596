%!TEX root = cloudcops.tex
\begin{abstract}
  As the cloud becomes the de-facto platform to deliver online
  services, it is becoming both an attractive target for attackers to
  disrupt hosted services, steal data, and compromise 
  resources to launch attacks on external targets.
  %Yet, very little is known about the prevalence of network-based
  %attacks from and towards cloud infrastructures. 
  In this paper, using three months
  of NetFlow data from a major cloud provider, we present the first
  large-scale characterization of attacks {\em inbound} and {\em outbound} the
  cloud.  We investigate ten types of attacks ranging from network-level DDoS attacks to application-level spam. Our results highlight the complexity, intensity, duration, and distribution of these
  attacks. Our results shed light on the key challenges and solutions for future attack prevention systems in the cloud.
%  Experience with the cloud provider suggests that existing
% defenses are not comprehensive and don't scale to the traffic
% volumes seen in our dataset. From our observation, we provide 
%implications and suggestions on attack detection and mitigation
%at cloud scale.
\end{abstract}

\section{Introduction}
% This paper presents a large-scale field study to characterize
% network-based attacks targeting the cloud
% as well as originating from hosted services/VMs under adversarial
% control.
Cloud services are growing rapidly and their market is expected to reach \$180 billion by 2015~\cite{Gartner}.
%
%Unfortunately, the cloud infrastructure and its hosted
%services are increasingly becoming an important
%target for attacks
Today, large cloud providers host tens of 
thousands of different services, so {\em inbound} attacks targeting the cloud can cause significant, and sometimes spectacular, collateral damage.
A recent survey of datacenter operators indicates that half of them
experienced DDoS attacks, with 94\% of those experiencing regular
attacks~\cite{arborworldwideinfrasecurityreport}. 
%\minlan{cite some arbor report about data center contribute a significant amount of the attacks in the Internet}
Moreover, attackers can abuse hosted services or compromised VMs in the
cloud to launch \emph{outbound} attacks
%in addition to hosting
%malware~\cite{googlesecurityreport,mysqlattack}, stealing confidential
%data~\cite{sonyattack}, 
%disrupting a competitor's service~\cite{darkreading}, and selling
%compromised VMs in the underground
%economy~\cite{caballero2011measuring,stone2011underground}.
%It has been reported that attackers have used cloud VMs 
e.g., to deploy
botnets~\cite{googlesecurityreport}, exploit kits to detect
vulnerabilities~\cite{grier12manufacturing}, send
spam~\cite{kanich11show,tasterchoice}, or launch DoS attacks to other
sites~\cite{booth13cloud}.
In April 2011, an attack on the Sony Playstation network compromising
more than 100 million customer accounts was carried out by a malicious
service hosted on Amazon EC2~\cite{sonyattack}.

According to the recent reports of individual attacks on enterprise and cloud
 networks~\cite{googlesecurityreport,Prolexic}, we highlight several unique features and technical trend
 of attacks in the cloud:
 1) Large-scale. These attacks have 
 the volume up to around 100 Gbps against a single cloud service. 
 2) High diversity of types. The attack types range from network-layer (e.g. SYN flood, UDP flood) to
 application-layer (e.g. HTTP GET, SQL injection) with different characteristics on volume, number of connection, packet header signatures. 
 3) Fast ramp-up rate. The attack traffic ramps up quickly and victimize the target cloud service usually within one minute.
This arises high challenges for attack detection and mitigation in the cloud to meet those requirements.

To detect attacks, cloud operators commonly deploy
commercial hardware boxes at the network, such as Firewalls, IDS, DDoS-protection appliances. 
However, Firewalls and IDS usually fail under high attack load. Even the specialized DDoS-protection appliances
(which focuses on volumetric attacks) is sometimes overwhelmed by the volume of attack
traffic seen by the cloud provider. Moreover, the DDoS-protection appliances has high monetary cost and inflexible progammability with third party software. Other software-based middleboxes (e.g. load balancers, OpenNF) can be specified to aid attack detection but they also cannot handle such high traffic load and it is still challenging to have quick detection relative to attack ramp-up rate. 
%
Prolexic~\cite{Prolexic} provides high-capacity private scrubbing networks but it also relies on fast and accurate attack detection. Further, it requires the traffic to be re-routed, which is not feasible for private tenant traffic in the cloud.
%
Google adopts an approach of absorbing DDoS bursts by killing low-priority jobs and pooling available servers. However, it is impossible to kill tenant instances in a multi-tenant cloud. 
%
None of these above systems provides comprehensive
coverage of the large attack diversity in the cloud. 

In this paper, we propose a new paradigm of  
{\em detection-as-a-service} and {\em mitigation-as-a-service}
that leverages commodity VMs for attack detection. 
%somewhat counter-intuitive solution: using
Our approach, instantiated in a service called \nimbus, combines the
elasticity of cloud computing resources 
with the kinds of programmability seen in software-defined networks
(SDN).
%
It allows us to scale resource usage with traffic demands, flexibility
to handle attack diversity, and resilience against volumetric attacks
designed to subvert the detection infrastructure.
%
\nimbus's central controller directs different traffic aggregates to
different VMs, each of which detects attacks destined to different
sets of cloud services.
%
Each VM can be easily programmed to detect the wide variety of attacks
discussed above, and when a VM is close to resource exhaustion, the
controller can divert some of its traffic to other, possibly newly
instantiated, VMs.
%
There are several key challenges in designing and implementing
\nimbus: including auto-scaling VMs to minimize traffic
redistributions, devising appropriate interfaces between the
controller and the VMs, and determining an efficient yet clean
functional separation between user and kernel-space processing for
traffic.
%
Our prototype built using servers with 10G links shows that \nimbus
can quickly scale-out virtual machines to analyze traffic at line
speed while providing reasonable accuracy for attack detection.




