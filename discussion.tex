\section{Discussion}

\parab{Robust to wise adversary.}
A wise adversary may be aware of our detection and mitigation system and generate sophisticated attacks to a) evade our \nimbus system; b) reduce the effectiveness of \nimbus system.
%
For example, the wise adversary may generate attack burst in a very short amount of time and repeat it periodically but with a relatively low average throughput. 
%
\nimbus may highly likely miss this attack from the use of traffic sampling. 
%
Another example, the wise adversary may become aware of the stage of mitigation. So he can subside the attack when \nimbus is scaled-up and launch it again when the system is about to scale-back. Doing so will cause fluctuation and decreasing efficiency in our system.
%
How to design a robust system to defense such a wise adversary is a challenge problem.

\parab{Mitigation-as-a-service.}
Once an attack being identified, the system needs to provide a set of functionalities to mitigate attacks.
%
Since the system can track the traffic changes, it can potentially identify the begin of an attack. 
%
For example, we can identify and mitigate an attack when the traffic starts to ramp-up and, as a result, minimize its damage to the target service. 
%
In this sense, we need to investigate when to start and stop the mitigation, which flow to mitigate so we achieve both effectiveness and low collateral damage to legitimate traffic, 
and how to adjust detection accuracy for fast attack detection but with low false positive. 
%
In general, we need to consider the balance of resource usage of detecting and mitigating attacks, detection accuracy, and collateral damage to legitimate traffic.

